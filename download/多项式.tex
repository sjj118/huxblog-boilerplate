\documentclass{beamer}
\usepackage[UTF8,noindent]{ctexcap}
\usetheme{Copenhagen}
\begin{document}
\title{多项式简介}
\author{史记}
\date{\today}

\begin{frame}
\titlepage
\end{frame}

\begin{frame}{系数表示法}
定义n次多项式为$A(x)=\sum_{i=0}^{n}a_ix^i=a_0+a_1x+a_2x^2+...+a_nx^n$\\
a是这个多项式的系数数组\\
最常用的表示多项式的方法就是用系数数组表示\\
\end{frame}

\begin{frame}{多项式的基本运算}
令A和B为两个n次多项式\\
加减:$(A\pm B)(x)=A(x)\pm B(x)=\sum_{i=0}^{n}(a_{i}\pm b_{i})x^{i}$\\
乘法:$(A\times B)(x)=\sum_{i=0}^{n}\sum_{j=0}^{n}a_{i}b_{j}x^{i+j}$
\end{frame}

\begin{frame}{多项式的点值表示法}
\begin{block}{定理}
已知平面上n个横坐标不同的点,则有且仅有1个n-1次多项式经过这n个点。
\end{block}
\pause
于是就有了另一种表示多项式的方法:用多项式曲线上n-1个点的坐标表示
\end{frame}

\begin{frame}{点值表示法与系数表示法的转换}
\begin{itemize}
\item 系数表示法->点值表示法:\\
随便取n+1个不同的数作为横坐标代进去\\
\pause
\item 点值表示法->系数表示法:\\
称为插值\\
常用算法是拉格朗日插值法,后面会介绍\\
\end{itemize}
\end{frame}

\begin{frame}{点值表示法下的基本运算}
设插值所用的横坐标为$x_0,x_1,...x_n$\\
L表示插值函数\\
$A(x)=L(A(x_0),A(x_1),...,A(x_n)),B(x)=L(B(x_0),B(x_1),...,B(x_n))$\\
\pause
加减:$(A\pm B)(x)=L((A\pm B)(x_0),(A\pm B)(x_1),...,(A\pm B)(x_n))=L(A(x_0)\pm B(x_0),A(x_1)\pm B(x_1),...,A(x_n)\pm B(x_n))$\\
\pause
乘法:$(A\times B)(x)=L((A\times B)(x_0),(A\times B)(x_1),...(A\times B)(x_n))=L(A(x_0)\times B(x_0),A(x_1)\times B(x_1),...,A(x_n)\times B(x_n))$
\end{frame}

\begin{frame}{快速傅立叶变换}
FFT可以在$O(nlogn)$的时间复杂度内从特定的n个横坐标上插值出多项式\\
准确的说是$\omega_n^1,\omega_n^2,...\omega_n^n$\\
$\omega_n$是复数,表示n次单位根\\
FFT的过程这里不深入探究\\
如果要计算两个系数表示的多项式的乘积的系数表示,就可以先转换成点值表示,然后相乘,再插值回去
\end{frame}

\begin{frame}{拉格朗日插值}
FFT只能从特定的横坐标插值,如果想要从任意点插值,就要使用拉格朗日插值法了\\

\end{frame}

\end{document}